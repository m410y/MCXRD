\documentclass[12pt]{article}
\usepackage[utf8]{inputenc}
\usepackage[russian]{babel}
\usepackage{amsmath}
\usepackage{amssymb}
\begin{document}

\section*{Модель точечного кристалла}

Дифракция от кристалла берется как от одной точки в пространстве,
обладающей всеми его общими свойствами при дифракции.
Применимость модели зависит от длины когерентности излучения,
размеров кристалла и его поглощения.
Ее преимущество состоит в геометрической простоте, и позволяет
учесть основные эффекты, вызваные геометрией установки.
Например: центрировка кристалла, смещение и наклоны осей гониометра,
расходимость пучка, изменение интенсивности от расстояния до его центра,
поворот и смещение детектора.

\subsection*{Основа модели}

Если на кристалл падает монохроматическое излучение с волновым вектором
$\pmb{k}$, интенсивностью $I_0$ и поляризацией $\pmb{e}$, то интенсивность
дифрагированного излучения с волновым вектором $\pmb{k}'$ в первом приближении будет
пропорциональна $I_0 (\frac{\pmb{k}'}{k'}, \pmb{e})^2 |f(\pmb{k}' - \pmb{k})|^2$, где
$f(\pmb{q})$ - фурье образ электронной плотности в кристалле, или, другими словами,
структурная амплитуда. При этом величины волновых векторов должны совпадать $k' - k = 0$.
Таким образом представив первичный пучок в виде суммы волн и задав $|f|^2$ можно
легко получить общее выражение для интенсивности.

Интенсивность первичного пучка будет описываться функцией
$I_0(\pmb{k}, \pmb{e})$, зависящей от
волнового вектора $\pmb{k}$ и поляризации $\pmb{e}$. Структурная амплитуда
задана и равна $F(\pmb{q})$, тогда интенсивность дифрагированного излучения:

\[I(\pmb{k}', \pmb{e}) = \int_{\mathbb{R}^3} I_0(\pmb{k}, \pmb{e}) F(\pmb{k}' - \pmb{k})
\left[\frac{\pmb{k}'}{k'} \times \pmb{e}\right]^2 \delta(k' - k) d^3 \pmb{k}\]

Далее представим $F(\pmb{q})$ в более привычном виде. Из-за трансляционной
симмтерии в заметно отлична от нуля структурная амплитуда будет только
в окрестностях узлов обратной решетки, целочисленные координаты которых
являются индексами Миллера $h, k, l$. Обозначим такой трехмерный целочисленный
вектор как $\pmb{h}$. Координаты узлов $\pmb{q_h}$в обратном пространстве тогда
представляются в виде:

\[\pmb{q} = 2 \pi (\pmb{a}^\ast h + \pmb{b}^\ast k +\pmb{c}^\ast l) = \hat{C}\pmb{h}\]

Матрица $C$ отличается от матрицы ориентации кристалла $UB$ только множителем $2 \pi$.
Теперь можно записать структурную амплитуду в виде:

\[F(\pmb{q}) = \sum_{\pmb{h} \in \mathbb{Z}^3} F_{\pmb{h}}(\pmb{q} - \hat{C} \pmb{h})\]

Где $F_{\pmb{h}}(\pmb{q})$ - функция, заметно отличная от нуля только в небольшой окресности,
размеры которой много меньше векторов трансляций обратной решетки.

Также бурем рассматривать близкий к монохроматическому и узконаправленный
первичный пучок. Тогда в свою очередь функция $I_0$ будет заметно
отлична от нуля только в окресности некого волнового вектора $\pmb{k}_0$.

Таким образом, рассматривая в первом приближении $F_{\pmb{h}}$ и $I_0$ как
дельта-функции можно получить известные общие уравнения, описывающие дифракцию на кристалле:

\[\begin{cases}
\pmb{k'} = \pmb{k}_0 + \hat{C}\pmb{h} \\
k' = k_0
\end{cases}\]

\subsection*{Измерение интенсивности}

Регистрация излучения происходит детекторами. При измерении они чаще всего остаются
неподвижными, а кристалл равномерно вращается со временем. Первым этапом
для описания детектора будет нахождение интенсивности излучения не просто
обладающего определенным волновым вектором, а проходящим через определенную
точку в пространстве. Другими словами, необходимо работать со спектрами
излучения, зависящими от координат.

Так как свет распространяется прямолинейно, то разница векторов между
конечной и начальной точками распространения луча должна быть сонаправленна
его же волновому вектору. В целом, это все работает за счет приближения
геометрической оптики, отлично описывающей работающей в нашем приближении.

Тогда, вводя зависимость спектра от координат мы вводим зависимость
интенсивности первичного пучка $I_0(\pmb{r}, \pmb{k}, \pmb{e})$ от
точки в пространстве $\pmb{r}$. Положение кристалла как точки будет
описываться вектором $\pmb{c}$, но надо не забывать, что он является
твердым телом и его повороты влияют на структурную амплитуду.
При повороте, описывающимся оператором $\hat{R}$ таким, что
$\pmb{r} \rightarrow \hat{R} \pmb{r}$, структурная амплитуда изменяется как
$F(\pmb{q}) \rightarrow F(\hat{R}^{-1} \pmb{q})$. И, наконец, положение детектора
будем задавать вектором $\pmb{d}$.

В таком случае при дифракции будет учитываться спектр первичного
пучка в точке $\pmb{r} = \pmb{c}$, где находится кристалл.
Направление же регистрируемого дифрагированного луча будет полностью
определятся положением кристалла и детектора, а именно
$\pmb{k}' \upuparrows \pmb{d} - \pmb{c}$.

Также так как детектором регистрируется вообще говоря все
длины волн и поляризации, то в общем случае надо ввести некторую весовую функцию
для каждой длины волны, поляризации и направления. Реально же, так как
используется крайне узкий спектр, то зависимость от всех параметров
практически отсутствует. Таким образом для получения суммарной регистрируемой
интенсивности в точке с хорошей точностью можно просто просуммировать
весь спектр, попадающий в определенную точку:

\[ I(\pmb{d}, \pmb{c}, \hat{R}) = \sum_{\eta,\pmb{h}}
    \int_{\mathbb{R}^3} I_0(\pmb{c}, \pmb{k}, \pmb{e}_\eta)
    F_{\pmb{h}}\left(\hat{R}^{-1}\left(\pmb{n} k - \pmb{k}\right) - \hat{C} \pmb{h}\right)
\left[\pmb{n} \times \pmb{e}_\eta\right]^2 d^3 \pmb{k}\]

Где $\pmb{n} = \frac{\pmb{d} - \pmb{c}}{|\pmb{d} - \pmb{c}|}$ - единичный ветктор вдоль направления
дифрагированного луча.

\subsection*{Приближение вблизи рефлекса}

Теперь допустим, что мы нашли такие параметры $\hat{R}$, $\pmb{c}$ и $\pmb{d}$,
в окрестности которых интенсивность заметно отлична от нуля. В таком случае
среди всех $F_{\pmb{h}}$ заметно отличено от нуля только одно слагаемое.
Соответствующий ему вектор $\pmb{h}$ будет рефлексом в отражающем положении.
Теперь посмотрим на вид интенсивности на детекторе при небольшом отклонении
параметров относительно этого положения.

Сразу оговорюсь, что не будет варьироваться вектор поляризации. Так как
в реальных системах если интенсивность все таки и зависит от поляризации,
то от длины волны и положения в пространстве эта зависимость крайне мала, так что
интенсивность с хорошей точностью можно разбить на произведение двух функций:
первая зависит только от координаты в пространстве и волнового вектора, а другая только
от поляризации.

\[ I_0(\pmb{c}, \pmb{k}, \pmb{e}) \rightarrow I_P(\pmb{e}) I_0(\pmb{c}, \pmb(k)) \]

Если для векторов $\pmb{c}$ и $\pmb{d}$ приращения - это произвольные вектора
$\delta \pmb{c}$ и $\delta \pmb{d}$, то для оператора $\hat{R}$ есть особенности,
так как он является оператором поворота. В частности важным является то, что
так как любое его малое приращение соответствует малому повороту вокруг
какой-либо оси, то этот оператор всегда превращает вектор в ортогональный ему.
Таким образом, если $\pmb{v} = \hat{R} \pmb{v}_0$, то
$\delta \pmb{v} = \delta \hat{R} \pmb{v}_0 = \delta \hat{R} \hat{R}^{-1} \pmb{v}$,
причем $(\delta \pmb{v}, \pmb{v}) = 0$.

В отражающем положении вектор, стоящий в $F_{\pmb{h}} = \pmb{0}$. При варьировании,
соответственно, он превратиться в свое приращение при малом отклонении. Используя это,
найдем выражение для интенсивности:

\[ I \approx \sum_\eta I_P(\pmb{e}_\eta) \int_{\mathbb{R}^3}
I_0(\pmb{c} + \delta \pmb{c}, \pmb{k} + \delta \pmb{k}) \cdot\]
\[\cdot F_{\pmb{h}}\left(\delta \hat{R}^{-1} (\pmb{n} k - \pmb{k}) +
\hat{R}^{-1}(\delta \pmb{n} k + \pmb{n} \delta k - \delta \pmb{k})\right)
\left[(\pmb{n} + \delta \pmb{n}) \times \pmb{e}_\eta\right]^2 d^3 \delta \pmb{k}\]
\[ \delta \pmb{n} = \frac{1}{|\pmb{d} - \pmb{c}|}
(\delta \pmb{d} - \pmb{n} (\pmb{n}, \delta \pmb{d})) -
\frac{1}{|\pmb{d} - \pmb{c}|} (\delta \pmb{c} - \pmb{n} (\pmb{n}, \delta \pmb{c}))\]
\[ \delta k = \frac{1}{k}(\pmb{k}, \delta \pmb{k}) \]

Теперь чтобы уменьшить громозкость выражений введем операторы, получающиеся в
процессе вычисления, и укажем их поведение и спектр. Первый естественно
возникающий оператор - это оператор приращения выражения

\[ (\pmb{n} \delta k - \delta \pmb{k}) =
    - \left( \delta \pmb{k} - \frac{\pmb{n} (\pmb{k}, \delta \pmb{k})}{k} \right)
= -\hat{J} \delta \pmb{k} \]

Собственные вектора оператора $\hat{J}$ довольно легко видеть - это вектор
$\pmb{n}$, которому соответствует собственное число
$1 - \left(\frac{\pmb{k}}{k}, \pmb{n}\right) = 1 - \cos{2\theta}$.
Другие собственные вектора находятся в плоскости, ортогональной $\pmb{k}$ и
все их собственные числа равны $1$. Видно, что этот оператор вырождается
только если вектора $\pmb{k} \parallel \pmb{n}$. Также достаточно просто получить и
обратный ему оператор:

\[ \hat{J}^{-1} \pmb{v} = \pmb{v} + \frac{\pmb{n}(\pmb{k}, \pmb{v})}
{k - (\pmb{k}, \pmb{n})} = \pmb{v} + \frac{\pmb{n}(\pmb{k}, \pmb{v})}{k(1 - \cos{2\theta})}\]

Другой опертор, который мы введем - это оператор проекции на плоскость, ортогональную
вектору $\pmb{n}$:

\[ \hat{N} \delta \pmb{d} = \delta \pmb{d} - \pmb{n} (\pmb{n}, \delta \pmb{d})\]

Этот операор является частным случаем оператора $\hat{J}$, если $\pmb{k} \parallel \pmb{n}$.
И его спектр также состоит из вектора $\pmb{n}$, для которого собственное число $0$, а также
плоскости, ортогональной $\pmb{n}$, для которой собственные числа равны $1$.

Далее сделаем важное замечание: интегрирование происходит по всем волновым векторам,
и так как он меняется мало относительно самого себя, то раскладывать выражение внутри
$F_{\pmb{h}}$ по линейным составляющим приращений справедливо, но сама интенсивность,
как и структурная амплитуда в зависимости от $\delta \pmb{k}$ меняется значительно.
Вообще говоря, это и будет первым порядком приближения интенсивности на детекторе.
Нулевым порядком является по сути дельта-функция интенсивности, расположенная
в точке, соответствующей условиям дифаркции. Вторым порядком будет уже учет
изменения интенсивности от изменения положения кристалла $\delta \pmb{c}$, 
как и изменение поляризационного множителя из-за $\delta \pmb{n}$. Таким образом,
искомое выражение можно записать как:

\[ I = \sum_\eta I_P(\pmb{e}_\eta) \left[ \pmb{n} \times \pmb{e}_\eta \right]^2
\int_{\mathbb{R}^3} I_0(\pmb{c}, \pmb{k} + \delta \pmb{k}) \cdot \]
\[ \cdot F_{\pmb{h}}\left(\hat{R}^{-1} \left(\hat{R} \delta \hat{R}^{-1} \pmb{q} +
k \hat{N} \frac{\delta \pmb{d} - \delta \pmb{c}}{|\pmb{d} - \pmb{c}|}\right) -
\hat{R}^{-1} \hat{J} \delta \pmb{k}\right) d^3 \delta \pmb{k}\]

Это выражение позволяет в первом приближении получить форму пика, при известных
функциях $I_0$ и $F_{\pmb{h}}$, но иногда оно не слишком удобно, так как
бывет, что характерный размер пика у структурный амплитуды гораздо меньше, чем у
спектра первичного пучка. В таком случае удобнее перейти к другим
координатам, а именно обозначить аргуемент $F_{\pmb{h}}$ как $\delta q$, и
выразить $\delta \pmb{k}$ через него:

\[ I = \sum_\eta I_P(\pmb{e}_\eta) \left[ \pmb{n} \times \pmb{e}_\eta \right]^2
\int_{\mathbb{R}^3} F_{\pmb{h}}(\delta \pmb{q}) \cdot \]
\[ \cdot I_0\left(\pmb{c}, \pmb{k} + k \hat{J}^{-1}\left(\hat{R} \delta \hat{R}^{-1} \frac{\pmb{q}}{k} +
\hat{N}\frac{\delta \pmb{d} - \delta \pmb{c}}{|\pmb{d} - \pmb{c}|} \right) -
\hat{J}^{-1} \hat{R} \delta \pmb{q}\right) \frac{d^3 \delta \pmb{q}}{1 - \cos{2\theta}} \]

\subsection*{Координаты детектора}

Пока мы считали детектор точечным и интенсивность измеряли в
произвольной точке. По большей части это было сделано ради общности утверждений,
но если переходить к реальным декткторам, то мы будем рассматривать матричные. На
таком детекторе вводится двумерная система координат $(X, Y)$, соответствующая
пикселям на нем. Координаты в пространстве, соответсвующие координатам детектора
можно в общем случае ввести как:

\[ \pmb{d} = \pmb{d}_0 + \pmb{d}_x X + \pmb{d}_y Y\]

\subsection*{Проведение измерений}

Зачастую интенсивность регистрируется не при неподвижном кристалле, а при
его равномерном вращении. В этом случае регистрируется интегральная интенсивность,
попадающая на каждый пиксель. Интенсивность, регистрируемая каждым пикселем пропорциональная
реальной интенсивности изулчения и времени, которое она действовала. Но вследствие
равномерного вращения кристалла это время оказывается пропорционально изменению угла поворота.

В свою очередь параметры $\hat{R}$ и $\pmb{c}$ оказываются в процессе измерения
зависимыми только от угла поворота, а значит в первом порядке можно считать, что
они пропорциональны малому отклонению угла поворота от идеального.

Таким образом логично определить этот угол из оператора поворота. И в таком случае
можно выразить малые приращения через него:

\[ \hat{R}(\omega) = e^{\hat{R}_a \omega} \hat{R}_0 \]
\[ \pmb{c} = \pmb{c}_0 + \hat{R} \pmb{\varepsilon} \]
\[ \delta \hat{R} = \hat{R}_a e^{\hat{R}_a \omega} \hat{R}_0 \delta \omega =
\hat{R}_a \hat{R} \delta \omega \]
\[ \delta \pmb{c} = \delta \hat{R} \pmb{\varepsilon} =
\hat{R}_a (\pmb{c} - \pmb{c}_0) \delta \omega = \pmb{u} \delta \omega \]

Подставляя это в последнюю формулу для интенсивности мы получим:

\[ I = \sum_\eta \frac{I_P(\pmb{e}_\eta)\left[ \pmb{n} \times \pmb{e}_\eta \right]^2}{1 - \cos{2\theta}}
\int_{\mathbb{R}^3} F_{\pmb{h}}(\delta \pmb{q}) \cdot \]
\[ \cdot I_0\left(\pmb{c}, \pmb{k} + k \hat{J}^{-1}\left(\hat{R}_a^{-1} \frac{\pmb{q}}{k} \delta \omega +
\hat{N}\frac{\delta \pmb{d} - \pmb{u} \delta \omega}{|\pmb{d} - \pmb{c}|} \right) -
\hat{J}^{-1} \hat{R} \delta \pmb{q}\right) d^3 \delta \pmb{q} \]

Распишем более подробнее второе слагаемое во втором аргументе $I_0$:

\[ \hat{R}_a^{-1} \frac{\pmb{q}}{k} \delta \omega + \hat{N}\frac{\delta \pmb{d} -
\pmb{u} \delta \omega}{|\pmb{d} - \pmb{c}|} =
\left(\hat{R}_a^{-1} \frac{\pmb{q}}{k} - \frac{\hat{N}\pmb{u}}{|\pmb{d} - \pmb{c}|}\right) \delta \omega + \]
\[ + \frac{\hat{N}\pmb{d}_x}{|\pmb{d} - \pmb{c}|} \delta X +
\frac{\hat{N}\pmb{d}_y}{|\pmb{d} - \pmb{c}|} \delta Y =
\pmb{v}_\omega \delta \omega + \pmb{v}_x \delta X + \pmb{v}_y \delta Y = \delta \pmb{v}\]

Можно переписать выражение в более красивом виде, а затем свести его к свертке:

\[ I(\delta \pmb{v}) = \sum_\eta \frac{I_P(\pmb{e}_\eta)
\left[ \pmb{n} \times \pmb{e}_\eta \right]^2}{1 - \cos{2\theta}}
\int_{\mathbb{R}^3} F_{\pmb{h}}(\delta \pmb{q}) 
I_0\left(\pmb{c}, \pmb{k} + k \hat{J}^{-1} \delta \pmb{v} -
\hat{J}^{-1} \hat{R} \delta \pmb{q}\right) d^3 \delta \pmb{q} = \]
\[ = \sum_\eta I_P(\pmb{e}_\eta)\left[ \pmb{n} \times \pmb{e}_\eta \right]^2
\int_{\mathbb{R}^3} \left(F_{\pmb{h}} \circ k \hat{R}^{-1}\right) (\pmb{s})
\left(\tilde{I_0} \circ k \hat{J}^{-1}\right) \left(\delta \pmb{v} - \pmb{s}\right) d^3 \pmb{s} =\]
\[ = P(\pmb{n}) \left[\left(F_{\pmb{h}} \circ k \hat{R}^{-1}\right) \ast
\left(\tilde{I_0} \circ k \hat{J}^{-1}\right)\right] (\delta \pmb{v})\]

Здесь сделаны замены:
\[P(\pmb{n}) = \sum_\eta
\frac{I_P(\pmb{e}_\eta)\left[ \pmb{n} \times \pmb{e}_\eta \right]^2}{1 - \cos{2\theta}}\]
\[ \tilde{I_0}(\pmb{v}) = I_0(\pmb{c}, \pmb{k} + \pmb{v}) \]
\[\pmb{s} = \hat{R} \frac{\delta \pmb{q}}{k}\]
Знак $\circ$ обозначает композицию функций (операторов).
Знак $\ast$ обозначает свертку, в данном случае - трехмерную.

Далее, как уже упонималось, для получения полной интенсивности при измерении
необходимо проинтегрировать эту интенсивность по углу $\delta \omega$. Интегрирование по
углу для воспроизводимости берется так, чтобы полностью покрыть всю облась, в которой
происходит дифракция. Таким образом можно спокойно ставить бесконечные пределы
при интегрировании:

\[ I_{rot}(\delta X, \delta Y) = \int_{\mathbb{R}} I(\pmb{v}_\omega \delta \omega +
\pmb{v}_x \delta X + \pmb{v}_y \delta Y) d \delta \omega\]



\end{document}
